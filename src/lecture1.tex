% TeX root = ../Main.tex

% First argument to \section is the title that will go in the table of contents. Second argument is the title that will be printed on the page.
\section[Lecture 1 (16 August 2023) -- {\it Linear Equations in Linear Algebra}]{Lecture 1 (Linear Equations in Linear Algebra)}

\subsection{Systems of Linear Equations}

\begin{definition}[Linear Equation]
    $a_1 x_1 + a_2 x_2 + \ldots + a_n x_n = b$

\end{definition}

\begin{definition}[A system of linear equations; Linear System]
    A collection of one or more linear equations involving the same set of variables, say, $x_1, x_2, \ldots, x_n$
    
\end{definition}

\begin{definition}[Solution of a Linear System]
    A list $\left( s_1, s_2, \ldots, s_n  \right)$ of numbers that makes each equation in the system true when the values $s_1, s_2, \ldots, s_n$ are substituted for $x_1, x_2, \ldots, x_n$ respectively.

    A \textbf{consistent} system has either one solution or infinitely many solutions

    An \textbf{inconsistent} system has no solution.
\end{definition}

\begin{theorem}
    A system of linear equations has (1) no solution, (2) exactly one solution, or (3) infinitely many solutions.
\end{theorem}

\begin{definition}[Solution set]
    The set of all possible solutions of a linear system.
\end{definition}

\begin{definition}[Equivalent systems]
    Two linear systems with the same solution set.
\end{definition}

\subsection{Vectors and Matrices}

